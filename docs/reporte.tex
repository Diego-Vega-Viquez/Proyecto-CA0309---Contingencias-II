\documentclass{article} % A4 paper and 11pt font size
%%%%%%%%%%%%%%%%%%%%%%%%%%%%%%%%%%%%%%%%%%%%%%%%%%%%%%%%%%%%%%%%%%%%%
% PAQUETES            											    %
%%%%%%%%%%%%%%%%%%%%%%%%%%%%%%%%%%%%%%%%%%%%%%%%%%%%%%%%%%%%%%%%%%%%%

\usepackage[spanish]{babel}
\linespread{1} % Interlineado
\usepackage[utf8]{inputenc} % Para poner notas al pie
\usepackage[hidelinks]{hyperref}
\usepackage{diegocon}
\usetikzlibrary{arrows.meta, positioning}

%%%%%%%%%%%%%%%%%%%%%%%%%%%%%%%%%%%%%%%%%%%%%%%%%%%%%%%%%%%%%%%%%%%%%
% Comandos          									  
%%%%%%%%%%%%%%%%%%%%%%%%%%%%%%%%%%%%%%%%%%%%%%%%%%%%%%%%%%%%%%%%%%%%%

\newcommand{\yo}{Diego Alberto Vega Víquez}
\newcommand{\yoCorto}{Diego Vega V.}
\newcommand{\profesor}{Eny Vargas Ulate}
\newcommand{\curso}{Contingencias de Vida II}
\newcommand{\siglaCurso}{CA-0309}
\newcommand{\grupo}{Grupo \#tal}

%%%%%%%%%%%%%%%%%%%%%%%%%%%%%%%%%%%%%%%%%%%%%%%%%%%%%%%%%%%%%%%%%%%%%
% Formateo de página            									  
%%%%%%%%%%%%%%%%%%%%%%%%%%%%%%%%%%%%%%%%%%%%%%%%%%%%%%%%%%%%%%%%%%%%%

% ------------------ MÁRGENES ------------------
\geometry{
    left=2.5cm,
    right=2.5cm,
    top=2.5cm,
    bottom=2.5cm
}

% ------------------ ENCABEZADOS PERSONALIZADOS ------------------
\fancypagestyle{miestilo}{ 
    \fancyhf{} 
    \fancyhead[L]{\grupo}
    \fancyhead[C]{Proyecto Final}
    \fancyhead[R]{\siglaCurso}
    \fancyfoot[L]{Prof. \profesor}
    \fancyfoot[R]{\thepage}
}

\begin{document}

% ------------------ PORTADA DEL DOCUMENTO ------------------
\begin{titlepage}
    \centering
    \vspace*{1cm}

    {\Large\bfseries Universidad de Costa Rica\par}
    \vspace{0.5cm}
    {\Large Escuela de Matemática\par}
    
    \vspace{2cm}
    
    {\Huge\bfseries Proyecto Final\par}
    \vspace{0.5cm}
    {\LARGE\siglaCurso : \curso\par}
    
    \vspace{2.5cm}
    
    {\large \grupo\par}
    \vspace{0.5cm}
    {\normalsize José Andrey Prado Rojas\par}
    {\normalsize \yo\par}

    \vspace{1.5cm}
    
    {\large Profesora:\par}
    \vspace{0.3cm}
    {\normalsize \profesor\par}
    
    \vfill
    
    {\normalsize \today\par}
\end{titlepage}

\clearpage
\begingroup
\begin{spacing}{1.05} % Aquí defines el interlineado
\fontsize{12pt}{15pt}\selectfont

\tableofcontents

\end{spacing}
\endgroup
\clearpage 

\newpage

\pagestyle{miestilo}

% ------------ RESUMEN ----------------
\section*{Resumen}
Breve descripción del propósito del informe, los métodos empleados (simulación de Monte Carlo y árboles binomiales), los resultados generales y conclusiones clave.

\newpage

% ---------------- INTRODUCCIÓN ----------------
\section{Introducción}
Describa el contexto institucional de Smart Investments S.A., los objetivos del análisis y la motivación de diversificar la cartera.
Mencione el propósito general del proyecto: evaluar la valoración y el riesgo del portafolio mediante simulación y analizar el impacto de una expansión accionaria.

% ---------------- METODOLOGÍA ----------------
\section{Metodología}
Explique la construcción de los modelos binomiales conjuntos para colones y dólares, la generación de 10,000 simulaciones de Monte Carlo y los cálculos de VaR y CVaR.
Incluya ecuaciones relevantes:
\[
FX(VaR_p) = \Pr[X \leq VaR_p] = p
\]
\[
CVaR_p(X) = E[X \mid X \geq VaR_p]
\]

% ---------------- RESULTADOS ----------------
\section{Resultados}
\subsection{Sección I: Valoración del portafolio}
Presente los valores esperados, pesimistas y optimistas del portafolio en los horizontes de 0, 3, 6, 12 y 15 años.
Incluya tablas y gráficos por moneda (colones y dólares).

\subsection{Sección II: Análisis de expansión accionaria}
Analice los resultados numéricos al año 15 bajo las estrategias: compra directa, forward, futuro, call europea y call americana.
Compare los valores finales del portafolio y comente las diferencias en riesgo y retorno.

% ---------------- CONCLUSIONES Y RECOMENDACIONES ----------------
\section{Conclusiones y recomendaciones}
Sintetice los principales hallazgos, conclusiones sobre el riesgo del portafolio y recomendaciones sobre la estrategia óptima para Smart Investments.

% ---------------- REFERENCIAS ----------------
\section*{Referencias}
\begin{itemize}
    \item Curvas Cero Cupón (Excel proporcionado por el curso)
    \item Material de clase: Valoración de Instrumentos Financieros I, II ciclo 2025
    \item Notas técnicas y bibliografía complementaria (Hull, Wilmott, etc.)
\end{itemize}

% ---------------- ANEXOS ----------------
\appendix
\section{Código fuente y simulaciones}
Incluya los scripts de R/Python o capturas de los resultados del notebook.
\section{Tablas adicionales}
Gráficos, distribuciones y comparaciones detalladas.

\end{document}

\end{document}