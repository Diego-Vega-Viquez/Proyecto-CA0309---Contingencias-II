\documentclass{article} % A4 paper and 11pt font size
%%%%%%%%%%%%%%%%%%%%%%%%%%%%%%%%%%%%%%%%%%%%%%%%%%%%%%%%%%%%%%%%%%%%%
% PAQUETES            											    %
%%%%%%%%%%%%%%%%%%%%%%%%%%%%%%%%%%%%%%%%%%%%%%%%%%%%%%%%%%%%%%%%%%%%%

\usepackage[spanish]{babel}
\linespread{1} % Interlineado
\usepackage[utf8]{inputenc} % Para poner notas al pie
\usepackage[hidelinks]{hyperref}
\usepackage{diegocon}
\usetikzlibrary{arrows.meta, positioning}

%%%%%%%%%%%%%%%%%%%%%%%%%%%%%%%%%%%%%%%%%%%%%%%%%%%%%%%%%%%%%%%%%%%%%
% Comandos          									  
%%%%%%%%%%%%%%%%%%%%%%%%%%%%%%%%%%%%%%%%%%%%%%%%%%%%%%%%%%%%%%%%%%%%%

\newcommand{\yo}{Diego Alberto Vega Víquez}
\newcommand{\yoCorto}{Diego Vega V.}
\newcommand{\profesor}{Eny Vargas Ulate}
\newcommand{\curso}{Contingencias de Vida II}
\newcommand{\siglaCurso}{CA-0309}
\newcommand{\grupo}{Grupo \#tal}

%%%%%%%%%%%%%%%%%%%%%%%%%%%%%%%%%%%%%%%%%%%%%%%%%%%%%%%%%%%%%%%%%%%%%
% Formateo de página            									  
%%%%%%%%%%%%%%%%%%%%%%%%%%%%%%%%%%%%%%%%%%%%%%%%%%%%%%%%%%%%%%%%%%%%%

% ------------------ MÁRGENES ------------------
\geometry{
    left=2.5cm,
    right=2.5cm,
    top=2.5cm,
    bottom=2.5cm
}

% ------------------ ENCABEZADOS PERSONALIZADOS ------------------
\fancypagestyle{miestilo}{ 
    \fancyhf{} 
    \fancyhead[L]{\grupo}
    \fancyhead[C]{Proyecto Final}
    \fancyhead[R]{\siglaCurso}
    \fancyfoot[L]{Prof. \profesor}
    \fancyfoot[R]{\thepage}
}

\begin{document}

% ------------------ PORTADA DEL DOCUMENTO ------------------
\begin{titlepage}
    \centering
    \vspace*{1cm}

    {\Large\bfseries Universidad de Costa Rica\par}
    \vspace{0.5cm}
    {\Large Escuela de Matemática\par}
    
    \vspace{2cm}
    
    {\Huge\bfseries Proyecto Final\par}
    \vspace{0.5cm}
    {\LARGE\siglaCurso : \curso\par}
    
    \vspace{2.5cm}
    
    {\large \grupo\par}
    \vspace{0.5cm}
    {\normalsize José Andrey Prado Rojas\par}
    {\normalsize \yo\par}

    \vspace{1.5cm}
    
    {\large Profesora:\par}
    \vspace{0.3cm}
    {\normalsize \profesor\par}
    
    \vfill
    
    {\normalsize \today\par}
\end{titlepage}

\clearpage
\begingroup
\begin{spacing}{1.05} % Aquí defines el interlineado
\fontsize{12pt}{15pt}\selectfont

\tableofcontents

\end{spacing}
\endgroup
\clearpage 

\newpage

\pagestyle{miestilo}

% ------------ RESUMEN ----------------
\section*{Resumen}
Breve descripción del propósito del informe, los métodos empleados (simulación de Monte Carlo y árboles binomiales), los resultados generales y conclusiones clave.

\newpage

% ---------------- INTRODUCCIÓN ----------------
\section{Introducción}
Describa el contexto institucional de Smart Investments S.A., los objetivos del análisis y la motivación de diversificar la cartera.
Mencione el propósito general del proyecto: evaluar la valoración y el riesgo del portafolio mediante simulación y analizar el impacto de una expansión accionaria.

% ---------------- METODOLOGÍA ----------------
\section{Metodología}

% \subsubsection*{Justificación de la tasa de descuento (NIC 19) y uso de la curva soberana del BCCR}

De conformidad con la NIC 19, la tasa de descuento debe estimarse con base en los rendimientos de mercado \emph{al cierre del ejercicio} para instrumentos de \emph{alta calidad} y en la \emph{misma moneda} y \emph{horizonte temporal} de las obligaciones. En economías donde no existe un mercado profundo de bonos corporativos de alta calidad en la moneda local, la práctica aceptada es utilizar la curva de rendimientos de \textbf{bonos soberanos} emitidos por el Gobierno en dicha moneda.

Para este trabajo, la Compañía reporta bajo NIC 19 con fecha de corte {31/12/2024}. En Costa Rica, dado que no existe un mercado profundo de bonos corporativos de alta calidad en colones, se adopta como referencia la {Curva de Rendimientos Soberana en colones} publicada por el \textit{Banco Central de Costa Rica} (BCCR), seleccionando el punto de la curva que mejor corresponda con la \emph{duración/promedio de vencimiento} de los pagos del plan.

Sea $t^\star$ la duración estimada (en años) del flujo de pagos de beneficios. Denotemos por $r_{\text{CRC}}(t)$ el rendimiento spot (cero cupón) de la curva soberana del BCCR en colones para plazo $t$. La tasa de descuento actuarial utilizada es entonces
\[
\text{Tasa de descuento} \;=\; r_{\text{CRC}}(t^\star)\quad \text{evaluada al 31/12/2024}.
\]
Operativamente:
\begin{enumerate}
    \item Se estima $t^\star$ a partir del calendario esperado de pagos del plan (jubilación, renuncia, fallecimiento e invalidez).
    \item Se consulta la Curva de Rendimientos Soberana en colones del BCCR (corte 31/12/2024) y se obtiene $r_{\text{CRC}}(t^\star)$.
    \item Si $t^\star$ no coincide con un nodo publicado, se interpola sobre la curva (lineal o mediante el método indicado por el BCCR) para obtener la tasa puntual.
\end{enumerate}

Esta elección cumple simultáneamente con: (i) \textbf{moneda} consistente (colones), (ii) \textbf{plazo} consistente (alineado con $t^\star$), y (iii) \textbf{fecha de medición al cierre} (31/12/2024), tal como requiere la NIC 19. La tasa resultante se emplea para descontar a valor presente los flujos de beneficios proyectados y para los análisis de sensibilidad (p.\,ej., $\pm 100$~pb).

% ---------------- RESULTADOS ----------------
\section{Resultados}
\subsection{Sección I: Valoración del portafolio}
Presente los valores esperados, pesimistas y optimistas del portafolio en los horizontes de 0, 3, 6, 12 y 15 años.
Incluya tablas y gráficos por moneda (colones y dólares).

\subsection{Sección II: Análisis de expansión accionaria}
Analice los resultados numéricos al año 15 bajo las estrategias: compra directa, forward, futuro, call europea y call americana.
Compare los valores finales del portafolio y comente las diferencias en riesgo y retorno.

% ---------------- CONCLUSIONES Y RECOMENDACIONES ----------------
\section{Conclusiones y recomendaciones}
Sintetice los principales hallazgos, conclusiones sobre el riesgo del portafolio y recomendaciones sobre la estrategia óptima para Smart Investments.

% ---------------- REFERENCIAS ----------------
\section*{Referencias}
\begin{itemize}
    \item Curvas Cero Cupón (Excel proporcionado por el curso)
    \item Material de clase: Valoración de Instrumentos Financieros I, II ciclo 2025
    \item Notas técnicas y bibliografía complementaria (Hull, Wilmott, etc.)
\end{itemize}

% ---------------- ANEXOS ----------------
\appendix
\section{Código fuente y simulaciones}
Incluya los scripts de R/Python o capturas de los resultados del notebook.
\section{Tablas adicionales}
Gráficos, distribuciones y comparaciones detalladas.

\end{document}

\end{document}